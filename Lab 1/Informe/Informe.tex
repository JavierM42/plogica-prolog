% !TeX spellcheck = <es_ANY>
\documentclass[]{article}
\usepackage[utf8]{inputenc}
\usepackage{pgfplots}
\usepackage{listings}
\usepackage{amsmath}

% Title Page
\title{Programación Lógica \\ Laboratorio 1 - Informe}
\author{
	Javier Morales\\
	4.780.020-9\\
	\and
	Javier Pérez\\
	4.697.236-2
	\and
	Melisa Techera\\
	5.138.335-4
}
\date{\today}


\begin{document}
\maketitle

\section{Eficiencia}

Todas las pruebas realizadas en una de las computadoras de facultad (Sala UDELAR D).

\subsection{Matrices I+J}

Se prueba la creación de una matriz NxN fija en un valor, y luego se aplica nuevo\_valor\_celda para cada fila y columna. Para generar el comando que invoca al cambio de valor se utilizó javascript, dado que es un número grande de llamados.

\begin{table}[h!]
	\begin{center}
		\caption{Eficiencia nuevo valor celda}
		\label{tab:table1}
		\begin{tabular}{l|r} % <-- Alignments: 1st column left, 2nd middle and 3rd right, with vertical lines in between
			\textbf{$N$} & \textbf{t(s)}\\
			\hline
			25 & 0.000 \\
			50 & 0.406 \\
			100 & 7.942 \\
			150 & -* \\
			200 & -* \\
		\end{tabular}
	\end{center}
\end{table}

\emph{*:} Error, out of local stack, demasiadas llamadas.
\newpage
\subsection{Cuadrado Latino}

Se prueba cuadrado latino con el predicado time/1, hasta el primer resultado.

\begin{table}[h!]
	\begin{center}
		\caption{Eficiencia cuadrado latino}
		\label{tab:table1}
		\begin{tabular}{l|r} % <-- Alignments: 1st column left, 2nd middle and 3rd right, with vertical lines in between
			\textbf{$N$} & \textbf{t(s)}\\
			\hline
			1 & 0.000 \\
			2 & 0.000 \\
			3 & 0.000 \\
			4 & 0.000 \\
			5 & 31.403 \\
			6 & $>$1000 \\		
		\end{tabular}
	\end{center}
\end{table}

Para $N=6$ la ejecución demora más de quince minutos. Conjeturamos que esto se debe a la naturaleza exponencial de las permutaciones (al ser $n!$ permutaciones, iterar sobre ellas tiene orden exponencial.) Latino hace múltiples llamadas a permutaciones, por lo que el aumento drástico es esperable.

\newpage
\subsection{Cuadrado Grecolatino}

Se prueba cuadrado grecolatino con el predicado time/1, hasta el primer resultado.

\begin{table}[h!]
	\begin{center}
		\caption{Eficiencia cuadrado grecolatino}
		\label{tab:table1}
		\begin{tabular}{l|r} % <-- Alignments: 1st column left, 2nd middle and 3rd right, with vertical lines in between
			\textbf{N} & \textbf{t(s)}\\
			\hline
			1 & 0.000 \\
			2 & 0.000* \\
			3 & 0.000 \\
			4 & 5.538 \\
			5 & 177.245 \\
			6 & $>$1000 \\		
		\end{tabular}
	\end{center}
\end{table}

\emph{*:} no se encontró solución (retorna false), porque no existen cuadrados grecolatinos de orden 2.

Para grecolatino el problema se agrava pues latino es invocado múltiples veces por él.





%\section{Ejercicio 1}
%
%Se relaja el problema a programación lineal encontrándose el óptimo 28/3 en $x_1=8/3$, $x_2=4/3$.
%
%Se obtiene la cota inferior $\underline{z}=28/3$. Como cota superior inicial se toma $\overline{z}=+\infty$.
%
%$x_1$ es una variable básica no entera en la solución (vale 8/3). Se tienen entonces dos subproblemas a resolver, uno agregando la restricción $x_1 \leq 2$ y otro agregando $x_1 \geq 3$.
%
%Por la primer rama, se encuentra la solución entera (2,2) con valor 10. Se ajusta la cota superior, $\overline{z}=10$.
%
%Por la segunda, se encuentra la solución no entera (3,5/4) con valor 39/4. Como está entre las cotas, no puede podarse. Se agregan entonces dos subproblemas, uno con $x_2 \leq 1$ y otro con $x_2 \geq 2$.
%
%Por la rama $x_2\leq1$, la solución óptima es (4,1) con valor 11, que no es mejor que la cota superior $\overline{z}$. Se poda por acotamiento.
%
%Por la rama restante, la solución óptima del problema relajado es (3,2) con valor 12, que también se poda por acotamiento. 
%
%La solución final está entonces en (2,2) con valor 10.
%
%
%
%
%\section{Ejercicio 2}
%
%\subsubsection*{Condiciones Lógicas}
%
%De (1), si $x_4=1$, $x_3$ también.
%
%\[ x_4 \leq x_3 \qquad (4) \]
%
%De (2), si $x_3=0$, todas las demás variables son cero. La última ecuación es repetida (4).
%
%\[ x_1 \leq x_3 \qquad (5) \]
%\[ x_2 \leq x_3 \qquad (6) \]
%
%De (3), si $x_4=1$, entonces $x_2=1$ y $x_3=1$. La segunda ecuación es repetida (4).
%
%\[ x_4 \leq x_2 \qquad (7) \]
%
%De (3), si $x_1=1$, el problema no es factible.
%
%\[ x_1 = 0 \qquad (8) \]
%
%De (2), si $x_2=1$ y $x_4=1$, el problema no es factible.
%
%\[ x_2 + x_4 \leq 1 \qquad (9) \]
%
%De (3), si $x_2=0$ y $x_3=0$ el problema no es factible.
%
%\[ x_2 + x_3 \geq 1 \qquad (10) \]
%
%De (6) y (10), si $x_3=0$ el problema no es factible.
%
%\[ x_3=1 \qquad (11) \]
%
%De (7) y (9) se deduce que si $x_4=1$ el problema no es factible.
%
%\[ x_4=0 \qquad (12)\]
%
%\subsubsection*{Resolución}
%
%Una vez fijas $x_1$, $x_3$ y $x_4$, es evidente que el conjunto factible del problema es ${(0,0,1,0), (0,1,1,0)}$.
%	
%Un conjunto minimal de restricciones es
%\[ x_1=0, x_3=1, x_4=0\].
%
%Como el término correspondiente a $x_2$ en la función objetivo es negativo, en (0,0,1,0) se alcanza el óptimo con un valor de 3.
%
%
%	
%\section{Ejercicio 3}
%
%\subsection*{a}
%
%Despejando $x_1$:
%
%\[ 5x_1 + 11x_2 \leq 29 \Rightarrow x_1 + \frac{11}{5}x_2 \leq \frac{29}{5} \] 
%
%Tomando función piso en $11/5$.
%
%\[ x_1 + 2x_2 \leq x_1 + \frac{11}{5}x_2 \leq \frac{29}{5} \]
%
%Como la valoración es entera, se toma función piso en $29/5$.
%
%\[ x_1 + 2x_2 \leq 5 \]
%
%Despejando $x_2$:
%
%\[ 5x_1 + 11x_2 \leq 29 \Rightarrow \frac{5}{11}x_1 + x_2 \leq \frac{29}{11} \]
%
%Aplicando función piso en los términos fraccionarios de la misma forma que anteriormente,
%
%\[ x_2 \leq 2 \]
%
%\subsubsection*{b}
%
%Siguiendo un procedimiento análogo al anterior (dividir por el coeficiente, luego aplicar función piso porque es una restriccion de menor o igual), despejando $x_1$:
%
%\[ x_1 \leq 2 \]
%
%Para $x_2$:
%
%\[ x_1 + x_2 \leq 5 \]
%
%\subsubsection*{c}
%
%Según las diapositivas del curso, el primer paso del procedimiento, para este problema, es:
%
%\[ \frac{1}{18}(5x_1 + 11x_2) + \frac{1}{18}(13x_1 + 7x_2) \leq \frac{29}{18} + \frac{37}{18} \]
%
%\[ x_1 + x_2 \leq \frac{66}{18} \]
%
%El segundo paso, aplicar función piso a los coeficientes de $x_i$ se saltea pues ya son enteros.
%
%Aplicando el tercer paso se obtiene la ecuación
%
%\[ x_1 + x_2 \leq 3 \]
%
%\subsubsection*{d}
%
%Analizando gráficamente las cinco ecuaciones obtenidas para el problema:
%
%\begin{tikzpicture}
%\begin{axis}[ 
%xtick = {0,1,...,3},
%ytick = {0,1,...,3},
%grid = both,
%xmin = 0,
%xmax = 3,
%ymin = 0,
%ymax = 3,
%xlabel=$x_1$,
%ylabel={$x_2$}
%] 
%\addplot[mark=none] {5/2 - x/2}; 
%\addplot[mark=none] {2}; 
%\addplot[mark=none] coordinates {(2, 0) (2, 3)};
%\addplot[mark=none] {5 - x}; 
%\addplot[mark=none] {3 - x}; 
%\addplot[mark=none, color=lightgray] {29/11 -5*x/11}; 
%\addplot[mark=none, color=lightgray] {37/7 -13*x/7}; 
%\addplot[mark=*] coordinates {(0,0)};
%\addplot[mark=*] coordinates {(1,0)};
%\addplot[mark=*] coordinates {(2,0)};
%\addplot[mark=*] coordinates {(0,1)};
%\addplot[mark=*] coordinates {(1,1)};
%\addplot[mark=*] coordinates {(1,2)};
%\addplot[mark=*] coordinates {(2,1)};
%\addplot[mark=*] coordinates {(0,2)};
%\end{axis}
%\end{tikzpicture}
%
%Se muestran las restricciones originales en gris y los puntos del casco convexo.
%
%El mismo queda determinado por 
%
%\[ x_1 + x_2 \leq 3 \]
%
%\[ x_1 \leq 2 \]
%
%\[ x_2 \leq 2 \]
%
%\newpage
%
%\section{Ejercicio 2}
%\subsection*{a}
%
%Analizando gráficamente las restricciones del problema queda determinado el conjunto de soluciones factibles.
%
%
%Si llamamos $f$ a la función objetivo, $f(x_1, x_2) = 2x_1 + x_2$, se obtienen los siguientes valores funcionales:
%
%\[ f(0,0) = 0 \]
%\[ f(1,0) = 2 \]
%\[ f(2,0) = 4 \]
%\[ f(0,1) = 1 \]
%\[ f(1,1) = 3 \]
%\[ f(2,1) = 5 \]
%\[ f(0,2) = 2 \]
%
%Mostrando claramente que $(2,1)$ es donde se alcanza el óptimo con un valor de $5$.
%
%\subsection*{b}
%
%Mirando el gráfico y conociendo cómo crece la función objetivo (más rápidamente en $x_1$ que en $x_2$), se obtiene el óptimo $46/7$ para $ x_1 = 18/7, x_2 = 6/7$.
%
%\subsection*{c}
%
%Un conjunto de restricciones que determina el casco convexo del problema es
%
%\[ x_1 \leq 2 \]
%\[ x_1 + 2x_2 \leq 4 \]
%
%\begin{tikzpicture}
%\begin{axis}[ 
%xtick = {0,1,...,3},
%ytick = {0,1,...,3},
%grid = both,
%xmin = 0,
%xmax = 3,
%ymin = 0,
%ymax = 3,
%xlabel=$x_1$,
%ylabel={$x_2$}
%] 
%\addplot[mark=none] coordinates {(2, 0) (2, 3)};
%\addplot[mark=none] {2 - x/2};  
%\addplot[mark=*] coordinates {(0,0)};
%\addplot[mark=*] coordinates {(1,0)};
%\addplot[mark=*] coordinates {(2,0)};
%\addplot[mark=*] coordinates {(0,1)};
%\addplot[mark=*] coordinates {(1,1)};
%\addplot[mark=*] coordinates {(2,1)};
%\addplot[mark=*] coordinates {(0,2)};
%\addplot[fill=lightgray,draw=none] coordinates {(0,0) (2,0) (2,1) (0,2) (0,0)};
%\end{axis}
%\end{tikzpicture}
%
%\section{Ejercicio 3}
%
%A través de análisis de casos de las $2^3$ posibilidades en cada caso, se determinan los conjuntos factibles.
%
%\[X_1 = X_2 = X_3=  \{(0,0,0), (0,0,1), (0,1,0), (1,0,0), (0,1,1)\}\]
%
%$P_2$ es el tetraedro de vértices $(0,0,0), (0,0,5/3), (0,5/2,0), (1,0,0)$, mientras que $P_3$ es también un tetraedro con vértices $(0,0,0), (0,0,3), (0,3/2,0), (1,0,0)$. Evidentemente, estos tetraedros no se encuentran uno incluido en el otro. Por otro lado, analíticamente:
%
%\[
%x_1 + x_2 \leq 1 \to 2x_1 + 2x_2 \leq 2 
%\]
%\[
%x_1 + x_3 \leq 1 \to 3x_1 + 3x_3 \leq 3
%\]
%Sumando esas ecuaciones se obtiene la condición de $P_2$, por lo que $x \in P_1 \to x \in P_2$, o sea $P_1 \subseteq P_2$. Además, sumando tres veces la primera restricción a la segunda, se obtiene
%
%\[
%3x_1 + 4x_2 + x_3 \leq 3
%\]
%
%Como $x_2$ es positiva, esa condición es más restrictiva que la condición de $P_3$, obteniéndose $P_1 \subseteq P_2$.
% 
%\section{Ejercicio 4}



\end{document}          


